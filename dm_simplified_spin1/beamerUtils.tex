\usepackage{xcolor}
\usepackage{pifont}
\usepackage{hyperref}
\usepackage{soul}
%\usepackage{ifthen}
\usepackage{datetime}
\usepackage{outlines}
\usepackage{zref-savepos}
\usepackage{tikz}
%\usepackage{tikz-feynman}
\usepackage[compat=1.0.0]{tikz-feynman}
%% \usetikzlibrary{external}
%% \immediate\write18{mkdir -p pgf-img}
%% \tikzexternalize[
%%   prefix=pgf-img/,
%%   system call={
%%     lualatex \tikzexternalcheckshellescape -halt-on-error -interaction=batchmode -jobname="\image" "\texsource" || rm "\image.pdf"
%%   },
%% ]
%\usepackage{feynmf}
%\usepackage{feynmp}
%\usepackage{feynmp-auto}
\usetikzlibrary{arrows}
\usetikzlibrary{calc}
\usepackage[pscoord]{eso-pic} % The zero point of the coordinate
                              % systemis the lower left corner of the
                              % page (the default) 


%%%%%%%%%%%%%%%%%%%%%%%%%%%%%%%%%%%%%%%%%%%%%%%%%%%%%%
%% New data format for title page 
\newdateformat{mydate}{\monthname[\THEMONTH] \twodigit{\THEDAY}, \THEYEAR}


%%%%%%%%%%%%%%%%%%%%%%%%%%%%%%%%%%%%%%%%%%%%%%%%%%%%%%
%% A few aliases for colors

%% Colors available:
%% black, blue, brown, cyan, darkgray, gray, green, lightgray, lime,
%% magenta, olive, orange, pink, purple, red, teal, violet, white,
%% yellow

%% For \textcolor 
\newcommand{\mytcol} [2]{\textcolor[RGB]{#1}{#2}}
%
\newcommand{\tcblack}[1]{\textcolor{black}{#1}} 
\newcommand{\tcred}  [1]{\textcolor{red}  {#1}} 
\newcommand{\tcblue} [1]{\textcolor{blue} {#1}} 
\newcommand{\tcgreen}[1]{\textcolor{green}{#1}} 
\newcommand{\tccyan} [1]{\textcolor{cyan} {#1}} 
\newcommand{\tcdred} [1]{\textcolor[RGB]{180,   0,   0}{#1}} 
\newcommand{\tcdblue}[1]{\textcolor[RGB]{  0,   0, 150}{#1}} 
%%
%% For \color 
\newcommand{\mycol}  [1]{\color[RGB]{#1}}
%
\newcommand{\cblack}{\color{black}} 
\newcommand{\cred}  {\color{red}  } 
\newcommand{\cblue} {\color{blue} } 
\newcommand{\cgreen}{\color{green}} 
\newcommand{\ccyan} {\color{cyan} } 
\newcommand{\cdred} {\color[RGB]{180,   0,   0}} 
\newcommand{\cdblue}{\color[RGB]{  0,   0, 150}} 

%% Note: to save the current color, use this sintax:
%%  \colorlet{saved}{.}
%%  \color{whatever you want}
%%  \color{saved}


%%%%%%%%%%%%%%%%%%%%%%%%%%%%%%%%%%%%%%%%%%%%%%%%%%%%%%
%% Colored strikethrough
\newcommand{\stcol}[2]{
  \newdimen\textlen
  \settowidth{\textlen}{#2}
  %\hspace*{-1ex}\rlap{#2}{\color{#1}\rule[0.5ex]{\textlen}{.6pt}}
  \hspace*{0ex}\rlap{#2}{\color{#1}\rule[0.5ex]{\textlen}{.6pt}}
}


%%%%%%%%%%%%%%%%%%%%%%%%%%%%%%%%%%%%%%%%%%%%%%%%%%%%%%
%% Underlined URL 
\newcommand{\uhref}[2]{\href{#1}{\tcblue{\underline{#2}}}}
\newcommand{\uhlink}[2]{\hyperlink{#1}{\tcblue{\underline{#2}}}}

%%%%%%%%%%%%%%%%%%%%%%%%%%%%%%%%%%%%%%%%%%%%%%%%%%%%%%
%% For basic math 
\def\numeval#1{\the\numexpr#1\relax}
\def\dimeval#1{\the\dimexpr#1\relax}


%%%%%%%%%%%%%%%%%%%%%%%%%%%%%%%%%%%%%%%%%%%%%%%%%%%%%%
%% My personal TAB
\makeatletter
% \zsaveposx is defined since 2011/12/05 v2.23 of zref-savepos
\@ifundefined{zsaveposx}{\let\zsaveposx\zsavepos}{}
\makeatother
\newcounter{hposcnt}
\renewcommand*{\thehposcnt}{hpos\number\value{hposcnt}}
\newcommand*{\SP}{% set position
  \stepcounter{hposcnt}%
  \zsaveposx{\thehposcnt s}%
  \xspace
}
\makeatletter
\newcommand*{\UP}[1]{% use previous position
  \zsaveposx{hpos#1u}%
  \zref@refused{hpos#1s}%
  \zref@refused{hpos#1u}%
  \kern\zposx{hpos#1s}sp\relax
  \kern-\zposx{hpos#1u}sp\relax
  \xspace
}
\makeatother


%%%%%%%%%%%%%%%%%%%%%%%%%%%%%%%%%%%%%%%%%%%%%%%%%%%%%%
%% My personal bullet list 
\newcommand{\bul}   {\1[\scalebox{0.5}{\color{BlueBody}\ding{108}}]} 
\newcommand{\sbul}  {\2[\scalebox{0.6}{\color{BlueBody}\ding{228}}]} 
\newcommand{\ssbul} {\3[\color{BlueBody}--]} 


%%%%%%%%%%%%%%%%%%%%%%%%%%%%%%%%%%%%%%%%%%%%%%%%%%%%%%
%% Fix coordinate point for later use
%% (e.g. connect with lines, arrows) 
\newcommand{\tikzmark}[1]{\ensuremath{\tikz[overlay,remember picture] \node (#1) {};}}


%%%%%%%%%%%%%%%%%%%%%%%%%%%%%%%%%%%%%%%%%%%%%%%%%%%%%%
%% ''Floating'' text box with ''anchors'' for lines, e.g. 
%%   \tikztext{9cm}{2.5cm}{red,rotate=-5}{tb}{\cblue\it My text\\Continues}
\newcommand{\tikztext}[5]{
  \tikz[overlay,remember picture] \draw (#1,#2) node[draw,#3] (#4) {#5};
}

%%%%%%%%%%%%%%%%%%%%%%%%%%%%%%%%%%%%%%%%%%%%%%%%%%%%%%
%% Arrow between two points (can be fixed points from
%% \tikzmark or \tikztext, or simply coordinates. E.g. 
%%   \tikzarrow{red,thick,out=10,in=180}{(node1)+(2pt,0.5ex)}{(node2.west)}
\newcommand{\tikzarrow}[3]{
  \tikz[overlay,remember picture,-latex] \draw[#1] #2 to #3;
}

%%%%%%%%%%%%%%%%%%%%%%%%%%%%%%%%%%%%%%%%%%%%%%%%%%%%%%
%% ''Floating'' text boxes (but \tikztext is better!)
%%  \placetbox{<horizontal pos>}{<vertical pos>}{<stuff>}
\newcommand{\placetbox}[3]{
  \setbox0=\hbox{#3} % Put <stuff> in a box
  \AddToShipoutPictureFG*{ % Add <stuff> to current page foreground
    \put(\LenToUnit{#1\paperwidth},\LenToUnit{#2\paperheight}){\vtop{{\null}\makebox[0pt][c]{#3}}}
  }
}


%%%%%%%%%%%%%%%%%%%%%%%%%%%%%%%%%%%%%%%%%%%%%%%%%%%%%%
%% Text with shadow
%%   \begin{tikzpicture}[remember picture,overlay]   
%%       \textshadow{4}{0}{This is important!}
%%   \end{tikzpicture} 
\newcommand\textshadow[4]{
    \node[black!30!white] at (#1em+0.3em,#2em-0.2em) {
        \scalebox{2}{\Huge #4} 
    };
    \node[#3] at (#1em-0.06em,#2em+0.04em) {
        \scalebox{2}{\Huge #4} 
    };
}

%%%%%%%%%%%%%%%%%%%%%%%%%%%%%%%%%%%%%%%%%%%%%%%%%%%%%%
%% Text with blured shadow
%%   \begin{tikzpicture}[remember picture,overlay]   
%%       \textblur{4}{0}{This is important!}
%%   \end{tikzpicture} 
\newcommand\textblur[4]{
    \newcommand\xoffset{0.3}
    \newcommand\yoffset{0.2}
    %% % Blur
    %% %\foreach \x in {-0.1,0.1} {
    %% \foreach \x in {0.10, 0.05} {
    %%     %\foreach \y in {-0.1,0.1} {         
    %%     %\foreach \y in {0.10, 0.05} {         
    %%         %\node[black!20!white] at (#1em+\xoffset em+\x em,#2em-\yoffset em-\y em) {
    %%         \node[black!20!white] at (#1em+\xoffset em+\x em,#2em-\yoffset em-\x em) {
    %%             \scalebox{2}{\Huge #4} 
    %%         };
    %%     %}
    %% }

    \node[black!05!white] at (#1em+0.48em,#2em-0.32em) {
        \scalebox{2}{\Huge #4} 
    };
    \node[black!10!white] at (#1em+0.42em,#2em-0.28em) {
        \scalebox{2}{\Huge #4} 
    };
    \node[black!15!white] at (#1em+0.36em,#2em-0.24em) {
        \scalebox{2}{\Huge #4} 
    };
    \node[black!20!white] at (#1em+0.30em,#2em-0.20em) {
        \scalebox{2}{\Huge #4} 
    };
    \node[black!25!white] at (#1em+0.24em,#2em-0.16em) {
        \scalebox{2}{\Huge #4} 
    };
    \node[black!30!white] at (#1em+0.18em,#2em-0.12em) {
        \scalebox{2}{\Huge #4} 
    };
    \node[black!35!white] at (#1em+0.12em,#2em-0.08em) {
        \scalebox{2}{\Huge #4} 
    };
    \node[black!40!white] at (#1em+0.06em,#2em-0.04em) {
        \scalebox{2}{\Huge #4} 
    };
    \node[#3] at (#1em-0.06em,#2em+0.04em) {
        \scalebox{2}{\Huge #4} 
    };
}
